\documentclass[../Dissertation.tex]{subfiles}
\begin{document}
\subsubsection{Historical Stock Data}

Historical stock data comes in \lstinline|csv| format.
To download from the Google Finance API, we must append some parameters to the URL \lstinline|http://www.google.com/finance/historical|.
\begin{center}
 \begin{tabular}{||c c c||} 
 \hline
 Description & Parameter & Example Argument \\ [0.5ex] 
 \hline\hline
 Stock Symbol & q & =GOOG\\ 
 \hline
 Date from which to start collecting data & startdate & =Aug+23\%2C+2016 \\
 \hline
 Download \lstinline|csv| format & output & =csv \\ [1ex] 
 \hline
\end{tabular}
\end{center}
To do so, we concatenate a parameter and its argument and separate each pair with an ampersand.
The resulting string is known as a \textit{query string}
Then we prefix this string with a question mark, which dermarcates the query string section in a URL~\cite{Wiki:QueryString}.

The full URL would look something like:
\begin{center}
\lstinline|http://www.google.com/finance/historical?q=GOOG&startdate=Aug+23%2C+2016&output=csv|
\end{center}


The \lstinline|csv| file takes the following schema:
\begin{center}
 \begin{tabular}{||c | c | c | c | c | c||} 
 \hline
 Date & Open & High & Low & Close & Volume \\ [0.5ex] 
 \hline
\end{tabular}
\end{center}
Since we are only concerned with the price change between the stock price at the end of the day and the end of the next day (see equation~\ref{eqn:returns}), we need only the \lstinline|Close| field.

\end{document}