\documentclass[../Dissertation.tex]{subfiles}
\begin{document}
\subsubsection{getOptions.java}

The \lstinline|getOptions.java| class takes just one input: \lstinline|String[] symbols|.
Here, we mainly use two methods: \lstinline|getJSONfromURL()| and \lstinline|getOptionsfromJSON()|. 

In our \lstinline|main| method we instantiate an array: \lstinline|optionsData[] options|.
This data type will store our options data.
The array is as long as the number of stock symbols in \lstinline|symbols|.

We then iterate through  each element of \lstinline|symbols| in a for-loop.
On each iteration, we concatenate the URL for our \lstinline|JSON| data.
Then we invoke \lstinline|getJSONfromURL()| using our URL as an input and afterwards invoke \lstinline|getOptionsfromJSON()| using our \lstinline|JSON| data as an input.

\paragraph{getJSONfromURL()}

Here, we use Google's \lstinline|Gson| library~\cite{gson:2017} and follow instructions submitted by user2654569~\cite{user2654569:2014} on Stack Overflow.
This method downloads the \lstinline|JSON| file as an \lstinline|InputStream|.
Using \lstinline|JsonParser|, we parse this into a \lstinline|JsonElement|, which we convert and return as a \lstinline|JsonObject|.

\paragraph{getOptionsfromJSON()}

The \lstinline|JsonObject| retains the hierarchical structure of a \lstinline|JSON| file and allows us direct access to objects, arrays and their elements via a key-value-pair mapping.
However, it is difficult to navigate this data unless you know the keys in advance.
This is our reasoning for storing these data in \lstinline|optionsData[]|: ease of access.

The first object we are interested in is \lstinline|expiry|:
\lstinputlisting[firstnumber=2,firstline=2,lastline=6]{./assets/example.json}
To access the values it contains we must first construct a new \lstinline|JsonObject| using \lstinline|expiry| as the key.
\begin{lstlisting}[firstnumber = 48]
JsonObject expiry = json.get("expiry").getAsJsonObject();
\end{lstlisting}
Now we must use further keys on this object to access the values for year, month and day-of-month.
At this point there are no more underlying objects, so we can store these values in Java strings.
\begin{lstlisting}[firstnumber = 49]
        String expiryYear = expiry.get("y").toString();
        String expiryMonth = expiry.get("m").toString();
        String expiryDayofMonth = expiry.get("d").toString();
\end{lstlisting}
These strings give us a date with which we are able to compute the number of days until the option expires.
We use the method \lstinline|getNumDaystoExpiry()| for this.

The rest of the data are contained in \lstinline|JSON| arrays.
These arrays contains a number of elements, each of which have differing strike prices and are as such priced accordingly, depending on how far in the money the strike price gets them.
The structure of a put or call element in the \lstinline|JSON| array looks like:
\lstinputlisting[firstnumber=8,firstline=8,lastline=11]{./assets/example.json}
where \lstinline|p| represents the price of the option. 

As before, to get at the values, we must use the necessary keys.
But doing so gives us the option price and strike price. 
Since there are so many puts in the array, simply iterate through each element and extract everything.
In \lstinline|Historic.java| and \lstinline|MonteCarlo.java|, however, we only use the parameters of the first put.

We now initialize \lstinline|optionsData|:
\begin{lstlisting}[firstnumber = 84]
        optionsData options  = new optionsData();
        options.setCallPrices(callPrices);
        options.setPutPrices(putPrices);
        options.setStrikePrices(strikePrices);
        options.setDaystoMaturity(NumDaystoExpiry);
\end{lstlisting}
then return the result to \lstinline|main()|. 

\paragraph{getNumDaystoExpiry}

We now parse our the \lstinline|expiry| values from some strings into a Java \lstinline|Date| object.
We follow instructions submitted by BalusC~\cite{BalusC:2010b} on Stack Overflow.
First we specify the format of our string: \lstinline|yyyy MM d|.
Then we parse it.
\begin{lstlisting}[firstnumber = 34,basicstyle=\fontsize{8}{10}\sffamily, escapeinside={@}{@}]
expiryDate = format.parse(expiryYear + " " + expiryMonth + " " + expiryDayofMonth);
\end{lstlisting}
Afterwards, we follow instructions submitted by jens108~\cite{jens108:2013} on Stack Overflow that detail how to find the number of days between two dates.
It is simply a two step process:
\begin{lstlisting}[firstnumber =40,basicstyle=\fontsize{8}{10}\sffamily, escapeinside={@}{@}]
long diff = expiryDate.getTime() - currentDate.getTime();
        int NumDaystoExpiry = (int) TimeUnit.DAYS.convert(diff, TimeUnit.MILLISECONDS);
\end{lstlisting}
We then return the result to \lstinline|getOptionsfromJSON|.

To summarise, the general procedure of \lstinline|getOptions.java| is as follows:
					\begin{algorithm}[H]
						\caption{Class: getOptions.java}
						\begin{algorithmic}[1]
						\Function{getOptions.java}{\lstinline|String[] symbols|}
							\label{class:getOptions}							
							\State Declare \lstinline|optionsData[] options|						
							\ForAll {\lstinline|String[] symbols|}				
							\State Declare \lstinline|optionsData options|			
							\State Concatenate URL string
							\State Download \lstinline|JSON| string
							\State Parse into \lstinline|JsonObject|
							\State Extract data
							\State Set \lstinline|optionsData options|
							\State \Return \lstinline|optionsData options|
							\EndFor
							\State \Return \lstinline|optionsData[] options|
						\EndFunction
						\end{algorithmic}
					\end{algorithm}

\paragraph{Black Scholes Implementation}

\lstinline|optionsData.java| gives us an object that facilitates the access of our options data (of which we have lots).
The second benefit is that we can define instance methods that use these data to perform certain calculations.

Specifically, we have implemented a private method called \lstinline|getBlackScholesOptionPrices()|. 
This class makes use of the \lstinline|distribution| package from the \lstinline|org.apache.commons.math3| library~\cite{Apache:Math3}.
Its purpose is to implement the Black Scholes formulas described in equations~\ref{eqn:callbs},~\ref{eqn:putbs}, and~\ref{eqn:bsd1d2}.
To invoke this method, we use the public methods \lstinline|getBlackScholesPut()| or \lstinline|getBlackScholesCall()|.

\end{document}