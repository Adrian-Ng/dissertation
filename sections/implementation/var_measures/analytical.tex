\documentclass[../Dissertation.tex]{subfiles}
\begin{document}
\subsubsection{Analytical.java}

This class makes use of the \lstinline|distribution| library from the \lstinline|apache.commons.math3| package~\cite{Apache:Math3}.
This is our implementation of the Analytical approach described in Section \ref{section:analytical}.
More specifically, this is an implementation of the Linear Model shown in equation ~\ref{eqn:linear}.
The output of this class is a vector of three VaR estimates: one for each of our variance and volatility measures.

\paragraph{Preparation}
Before we can perform the necessary calculations, we must first prepare our variables, of which there are a few.
Notably, we first initialize a standard Gaussian variable:
\begin{lstlisting}[firstnumber=14]
	NormalDistribution distribution = new NormalDistribution(0,1);
\end{lstlisting}
We use the \lstinline|inverseCumulativeProbability()| method to find our percentile as described in equation~\ref{eqn:definepercentile}:
\begin{lstlisting}[firstnumber = 15]
	double riskPercentile = - distribution.inverseCumulativeProbability(1-p.getConfidenceLevel());
\end{lstlisting}
where \lstinline|p.getConfidenceLevel()| is our confidence level $c$.

Additionally, we initialize the array \lstinline|stockDelta| and populate it from \lstinline|Parameters p| using the method \lstinline|getStockDelta()|.
And for convenience, we initialize a one-dimensional array \lstinline|double[] currentStockPrices| for our current stock prices  and populate it from \lstinline|stockPrices|, which is a two-dimensional array.

The process for \lstinline|Analytical.java| continues as follows:
					\begin{algorithm}[H]
						\caption{Class: Analytical.java}
						\begin{algorithmic}[1]
						\Function{Analytical.java}{\lstinline|Parameters p, double[][] stockPrices|}
							\label{class:Analytical}		
							\State \lstinline|String[] volatilityMeasures| $\gets$ \lstinline|{EW, EWMA, GARCH}|
							\State \lstinline|double[][] priceChanges| $\gets$ \lstinline|stockPrices.getPercentageChanges()|
								\ForAll {i in volatilityMeasures}	
									\State \lstinline|correlationMatrix| $\gets$ \lstinline|priceChanges.getCorrelationMatrix(i+1)|
									\State \lstinline|volatility| $\gets$ \lstinline|priceChanges.getVolatility(i+1)|
									\State Compute Sum of Linear Components $\to$ \lstinline|sum|
									\State \lstinline|VaR| $\gets$ \lstinline|Math.sqrt(p.getTimeHorizon())|$\times$\lstinline|riskPercentile|$\times$\lstinline|Math.sqrt(sum)|
								\EndFor
							\State \Return \lstinline|VaR|
						\EndFunction
						\end{algorithmic}
					\end{algorithm}

\iffalse
\paragraph{Computation}

Now we perform the necessary calculations using methods we described in Section \ref{section:stats}.
First we compute our price changes \lstinline|doube[][] priceChanges| using \lstinline|getPercentageChanges()|.

Then we compute our correlation matrices.
We have three ways of estimating variances and volatilities, so we compute three matrices in a for-loop.
\begin{lstlisting}[firstnumber = 33]
correlationMatrix[i] = new Stats(priceChanges).getCorrelationMatrix(i+1);
\end{lstlisting}

Then for each stock variable, we compute the vector \lstinline|volatility| of three volatility measures using \lstinline|getVolatility(int measure)|.
Alternatively, we could replace the variables \lstinline|correlationMatrix| and \lstinline|volatility| with some covariance matrix.
We would simply use \lstinline|getCovarianceMatrix(int measure)|. 
The result would be the same and we would have one fewer variable.

Nonetheless, at this point, we have all the necessary components of the Linear Model.
We can now compute the sum of these components.
To do this, we use two nested for-loops in which we iterate through each element in our three correlation matrices.
At each element, we sum the following:
\begin{lstlisting}[firstnumber=48,basicstyle=\fontsize{8}{10}\sffamily, escapeinside={@}{@}]
sum[k] += stockDelta[i] * stockDelta[j] @\newline@* currentStockPrices[i] * currentStockPrices[j] @\newline@* correlationMatrix[k][i][j] @\newline@* volatility[k][i] * volatility[k][j];
\end{lstlisting}
where \lstinline|sum| is a vector of three elements. Now, for each of these three elements, we calculate VaR:
\begin{lstlisting}[firstnumber=51,basicstyle=\fontsize{8}{10}\sffamily, escapeinside={@}{@}]
VaR[i] = Math.sqrt(p.getTimeHorizon()) * riskPercentile @\newline@* Math.sqrt(sum[i]);
\end{lstlisting}
\lstinline|Analytical.java| is now finished and we return \lstinline|VaR|.
\fi
\end{document}