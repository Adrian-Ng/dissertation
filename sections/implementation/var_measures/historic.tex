\documentclass[../Dissertation.tex]{subfiles}
\begin{document}
\subsubsection{Historic.java}
\label{section:javahistoric}
This is our implementation of the Historical VaR measure as described in Section~\ref{section:historical}.
Unlike some other measures we implement, it returns only a single estimate of VaR.

\paragraph{Preparation}

We must first get some data from \lstinline|Parameters p|.
In doing so, we initialize \lstinline|int[] stockDelta| and \lstinline|int[] optionDelta|.

Then we declare some variables.
\begin{lstlisting}[firstnumber=21]
        double[] currentStockPrices = new double[numSym];
        double[][] strikePrices = new double[numSym][];
        double[][] currentPutPrices = new double[numSym][];
        int[] daystoMaturity = new int[numSym];
        double todayPi = 0;
\end{lstlisting}
These will hold current stock prices, the parameters of our put options and today's value of our portfolio.
We iterate through each of our stock symbols in a for-loop and populate these in the following way:
\begin{lstlisting}[firstnumber=28, escapeinside={@}{@}]
            currentStockPrices[i] = stockPrices[i][0];
            strikePrices[i] = options[i].getStrikePrices();
            daystoMaturity[i] = options[i].getDaystoMaturity();
            currentPutPrices[i] = options[i].getPutPrices();
            int numPuts = currentPutPrices[i].length;
            todayPi += stockDelta[i] * currentStockPrices[i] @\newline@+ optionDelta[i] * currentPutPrices[i][numPuts-1];
\end{lstlisting}

The process for \lstinline|Historic.java| continues as follows:
					\begin{algorithm}[H]
						\caption{Class: Historic.java}
						\begin{algorithmic}[1]
						\Function{Historic.java}{\lstinline|Parameters p, double[][] stockPrices, optionsData[] options|}
						\label{class:Historic}									
						\State \lstinline|double[][] priceChanges| $\gets$ \lstinline|stockPrices.getPercentageChanges()|
						\State \lstinline|tomorrowStockPrices| $\gets$ \lstinline|priceChanges| $\times$ \lstinline|currentStockPrices|
						\State \lstinline|tomorrowPutPrices| $\gets$ \lstinline|options.getBlackScholesPut(tomorrowStockPrices)|
						\State Revalue Portfolio $\to$ \lstinline|tomorrowPi|
						\State Sort \lstinline|tomorrowPi| Ascending
						\State \lstinline|index = (1-p.getConfidenceLevel())|$\times$\lstinline|tomorrowPi.length|
						\State \lstinline|VaR =(todayPi - tomorrowPi[index])|$\times$\lstinline|Math.sqrt(p.getTimeHorizon())|
						\State \Return \lstinline|VaR|
						\EndFunction
						\end{algorithmic}
					\end{algorithm}
\iffalse
\paragraph{Computation}

Now we perform the necessary calculations using methods we described in Section \ref{section:stats}.
First we compute our price changes \lstinline|double[][] priceChanges| using \lstinline|getPercentageChanges()|.

Then we iterate through all of our price changes, building up a distribution of tomorrow's stock prices using today's current stock variables \lstinline|currentStockPrices| and \lstinline|stockDelta|.
To perform this iteration, we use a nested for-loop, at each step going through every element in \lstinline|priceChanges|.
At each element, we compute:
\begin{lstlisting}[firstnumber=42,basicstyle=\fontsize{8}{10}\sffamily, escapeinside={@}{@}]
	tomorrowStockPrices[i][j] = (priceChanges[i][j] * currentStockPrices[i])@\newline@ + currentStockPrices[i];
\end{lstlisting}
	
We then use these stock price predictions to price tomorrow's put options. 
Here, we make use of \lstinline|getBlackScholesPut()|, which is a method defined in \lstinline|optionsData.java|.
Just as before, we use a nested for-loop but this time we iterate through every element in a different variable: \lstinline|tomorrowStockPrices|.
At each element, we compute:
\begin{lstlisting}[firstnumber=57,basicstyle=\fontsize{8}{10}\sffamily, escapeinside={@}{@}]
	tomorrowPutPrices[i][j] @\newline@= options[i].getBlackScholesPut(tomorrowStockPrices[i][j]);
\end{lstlisting}

Now we can revalue our portfolio using our predictions for tomorrow's market variables.
First we declare the one-dimensional array \lstinline|double[] tomorrowPi|.
Then, using a for-loop, we iterate through each our predictions, computing the combined value of all assets that contribute to our portfolio.
At each iteration, we use another loop in which we loop through each stock symbol and compute the predicted value of tomorrow's portfolio value as follows:
\begin{lstlisting}[firstnumber=52,basicstyle=\fontsize{8}{10}\sffamily, escapeinside={@}{@}]
for (int j = 0; j < numSym; j++)
                sum += (tomorrowStockPrices[j][i] * stockDelta[j]) @\newline@+ (tomorrowPutPrices[j][i] * optionDelta[j]);
\end{lstlisting}
where \lstinline|sum| is tomorrow's predicted portfolio value for this iteration.
Lastly, we set \lstinline|tomorrowPi[i]= sum| at the end of each iteration.

Once we have a full distribution of tomorrow's portfolio value in \lstinline|tomorrowPi|, we sort the array.
We use the \lstinline|Arrays.sort()| method, which sorts arrays in order of ascending value.
Therefore, the index of our desired cut-off point for a given confidence level, is computed as follows:
\begin{lstlisting}[firstnumber=58,basicstyle=\fontsize{8}{10}\sffamily, escapeinside={@}{@}]
double index = (1-p.getConfidenceLevel())*tomorrowPi.length;
double VaR = (todayPi - tomorrowPi[(int) index]) @\newline@* Math.sqrt(p.getTimeHorizon());
\end{lstlisting}
In the last step, we compute the increment $\Delta\Pi$ and multiply by the square root of our time horizon.
At this point we reach the end of \lstinline|Historic.java| and we return \lstinline|VaR|.
\fi
\end{document}