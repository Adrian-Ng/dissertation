\documentclass[../Dissertation.tex]{subfiles}
\begin{document}
\subsubsection{MonteCarlo.java}

This is our implementation of the Monte Carlo VaR measure as described in Section~\ref{section:montecarlo}.
In \lstinline|MonteCarlo.java|, we produce three estimates for VaR.
This is because we have three variance and volatility measures.
In addition to the \lstinline|main()| method, we will define here two private methods \lstinline|randomWalk| and \lstinline|weinerProcess|.
Both are integral parts to simulating a so-called random walk.

\paragraph{Preparation}

We begin in identical fashion to \lstinline|Historic.java| in which we get some data from \lstinline|Parameters p|, initialize some arrays and calculate the current value of the portfolio.
Please refer to~\ref{section:javahistoric} for more detail as we will skip the parts that are identical.

We first initialize some variables that are key to the Monte Carlo simulation.
We initialize \lstinline|int N = 24;| and \lstinline|int paths = 10000;|.
These are the number of steps in each random walk and the number of random walks respectively.
We also initialize \lstinline|double dt = 1/N;| which gives us time steps 1 hour in size.

The procedure for \lstinline|MonteCarlo.java| continues as follows:
					\begin{algorithm}[H]
						\caption{Class: MonteCarlo.java}
						\begin{algorithmic}[1]
						\Function{\lstinline|MonteCarlo.java|}{\lstinline|Parameters p,double[][] stockPrices,optionsData[] options|}
						\label{class:MonteCarlo}									
						\State \lstinline|int N| $\gets 24$
						\State \lstinline|int paths| $\gets 10000$
						\State \lstinline|double dt| $\gets 1/N$
						\State \lstinline|String[] volatilityMeasures| $\gets$ \lstinline|{EW, EWMA, GARCH}|
						\State \lstinline|double[][] priceChanges| $\gets$ \lstinline|stockPrices.getPercentageChanges()|
						\ForAll {i in volatilityMeasures}
							\State \lstinline|choleskyDecomposition[i]| $\gets$ \lstinline|priceChanges.getCholeskyDecomposition(i+1)|
							\ForAll {j in paths}
								\State \lstinline|tomorrowStockPrices[j]|$\gets$\lstinline|randomWalk(choleskyDecomposition)|
							\EndFor
							\State \lstinline|tomorrowPutPrices| $\gets$ \lstinline|options.getBlackScholesPut(tomorrowStockPrices)|
							\State Revalue Portfolio $\to$ \lstinline|tomorrowPi|
							\State Sort \lstinline|tomorrowPi| Ascending
							\State \lstinline|index = (1-p.getConfidenceLevel())|$\times$\lstinline|tomorrowPi.length|
							\State \lstinline|VaR =(todayPi - tomorrowPi[index])|
						\EndFor 
						\State \Return \lstinline|VaR|						
						\EndFunction
						\end{algorithmic}
					\end{algorithm}
					
\iffalse
\paragraph{Computation}

We compute the price changes of our stock prices in the usual way with \lstinline|getPercentageChanges()| and initialize \lstinline|double[][] priceChanges|.
Because we have three measures of variance and volatility, we build three matrices representing the Cholesky Decomposition of the variance-covariance matrix $\Sigma$.
To do this, we simply initialize our variable \lstinline|choleskyDecomposition| with the following constructor:
\begin{lstlisting}[firstnumber=77,basicstyle=\fontsize{8}{10}\sffamily, escapeinside={@}{@}]
	choleskyDecomposition[i] = new Stats(priceChanges).getCholeskyDecomposition(i + 1);
\end{lstlisting}
where \lstinline|int i| is our for-loop iterator.
Then we simulate tomorrow's stock prices via Monte Carlo simulation by invoking the \lstinline|simulatePath()| method.
We use a nested for-loop because for each Cholesky Decomposition, we must generate 10,000 random walks.
In doing so, we build three very large distributions for tomorrow's stock prices.
Below is a snippet of code illustrating the invokation of the method \lstinline|simulatePath()| at each point in the nested loop.
\begin{lstlisting}[firstnumber=85,basicstyle=\fontsize{8}{10}\sffamily, escapeinside={@}{@}]
double[] tuplePercentageChanges @\newline@= simuluatePath(N, @\newline@currentStockPrices, @\newline@dt, @\newline@choleskyDecomposition[i]);
\end{lstlisting}
Using these stock prices, we are able to price tomorrow's options.
We iterate through every stock price in our distributions and produce an option price for each.
This is done using the same method as we used in the Historical simulation in section~\ref{section:javahistoric}: \lstinline|getBlackScholesPut()|.
\begin{lstlisting}[firstnumber=94,basicstyle=\fontsize{8}{10}\sffamily, escapeinside={@}{@}]
                    tomorrowPutPrices[i][j][k] = options[j].getBlackScholesPut(tomorrowStockPrices[i][j][k]);
\end{lstlisting}
Now we have predictions for all of tomorrows market variables, we initialize \lstinline|double[][] tomorrowPi| and revalue the portfolio for all of tomorrow's predictions.
What follows is identical in nature to the Historical simulation, so we do not go in to great detail.
We loop three times through \lstinline|double[] tomorrowPi| and do as follows:
\begin{lstlisting}[firstnumber=107,basicstyle=\fontsize{8}{10}\sffamily, escapeinside={@}{@}]
            Arrays.sort(tomorrowPi[i]);
            double index = (1 - p.getConfidenceLevel()) * tomorrowPi[i].length;
            VaR[i] = (todayPi - tomorrowPi[i][(int) index]) * Math.sqrt(p.getTimeHorizon());        
\end{lstlisting}
We sort our distributions in ascending order and find the cut-off point for our confidence level.
Then, we calculate the increment $\Delta\Pi$ and multiply by the square root of our time horizon for our estimate of VaR.
\lstinline|MonteCarlo.java| ends and we return \lstinline|VaR|, which this time contains three estimates of VaR.
\fi
\paragraph{randomWalk()}

This method is our implementation of equation~\ref{eqn:montecarlo}, the stochastic process describing the Monte Carlo simulation.
It takes four parameters: \lstinline|int N|, the number of steps; \lstinline|double dt|,  he magnitude of each step; \lstinline|double[] currentStockPrices| and \lstinline|double[][] choleskyDecomposition|.

Monte Carlo is a discrete process, so we declare a grid in which we simulate our random walks:
\begin{lstlisting}[firstnumber = 27]
	double[][] grid = new double[numSym][N];
\end{lstlisting}
The starting values at the beginning of each walk will be our current stock prices.
So for \lstinline|N| steps, we build some random walks, one for each stock symbol.
We generate a vector of correlated random variables that take the multivariate Gaussian distribution $N\sim\phi(0,L)$, where $L$ is represented by the variable \lstinline|choleskyDecomposition|.
We do this in the following way:
\begin{lstlisting}[firstnumber=107,basicstyle=\fontsize{8}{10}\sffamily, escapeinside={@}{@}]
	grid[j][i] = (correlatedRandomVariables[j] @\newline@* grid[j][i-1] * Math.sqrt(dt)) + grid[j][i-1];
\end{lstlisting}
where \lstinline|correlatedRandomVariable[j]| represents our vector of correlated random variables at step \lstinline|j|.
This is initialized by the method \lstinline|weinerProcess()|.
When we have reached the end of the grid, we return the terminal stock price for each stock symbol via: \lstinline|terminalStockPrice[i] = grid[i][N-1];|.

\paragraph{weinerProcess()}

The Weiner Process is the $dz = \epsilon\sqrt{dt}$ part of equation~\ref{eqn:montecarlo}, where $\epsilon$ is the standard Gaussian variable.
We intialize the following to simulate our Gaussian variable.
\begin{lstlisting}[firstnumber = 9]
Random epsilon = new Random();
\end{lstlisting}
For each of our stock symbols, we sample from this variable and populate an array that is IID.
\begin{lstlisting}[firstnumber = 14]
	dz[i] = epsilon.nextGaussian();
\end{lstlisting}
Then we perform the matrix multiplication of \lstinline|double[] dz| and \lstinline|double[][] choleskyDecomposition|, which gives us \lstinline|double[] correlatedRandomVariables|.
\begin{lstlisting}[firstnumber = 17]
for(int i = 0; i < numSym; i++) {
            double sum = 0;
            for (int j = 0; j < numSym; j++)
                sum += choleskyDecomposition[i][j] * dz[j];
            correlatedRandomVariables[i] = sum;
\end{lstlisting}
Then we reach the end of \lstinline|weinerProcess()| and return \lstinline|correlatedRandomVariables|.

\end{document}