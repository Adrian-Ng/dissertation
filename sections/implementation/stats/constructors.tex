\documentclass[../Dissertation.tex]{subfiles}
\begin{document}
\subsubsection{Constructors}
\paragraph{\lstinline|double[] singleVector|}

For instance, we have a method called \lstinline|getMean()|.
It takes a single variable - a \lstinline|double[]| - and returns a \lstinline|double|.
That is, suppose we have an array representing a single vector of price changes: \lstinline|double[] priceChanges|.
To calculate the mean, we invoke the desired constructor and call the method as follows:
\begin{center}
\lstinline|double mean = new Stats(priceChanges).getMean();|
\end{center}
Because we only used a single \lstinline|double[]|, this means the following constructor was used:
\lstinputlisting[firstnumber=17,firstline=17,lastline=20]{"./assets/Stats.java"}
 
\paragraph{\lstinline|double[] xVector, double[] yVector|}

On the other hand, we might want to compute the covariance between a pair of variables, both \lstinline|double[]|.
For this, we use a method called \lstinline|getVariance()| which, this time takes, two variables of the type \lstinline|double[]| and returns a \lstinline|double|.
We compute the variance in the following fashion:
\begin{center}
\lstinline|double covariance = new Stats(priceChanges1, priceChanges2).getVariance();|
\end{center}
This time the constructor used looks like 
\lstinputlisting[firstnumber=21,firstline=21,lastline=22]{"./assets/Stats.java"}

\paragraph{\lstinline|double[][] multiVector|}

Let us also consider, for instance, the case when we need to compute the covariance matrix of some multivariate distribution.
In the case of our multivariate price changes, we would represent such a distribution with a \lstinline|double[][]|.
We use a method called \lstinline|getCovarianceMatrix()| which is called in the following way:
\begin{center}
\lstinline|double[][] covarianceMatrix = new Stats(priceChanges).getCovarianceMatrix();|
\end{center}
where \lstinline|priceChanges| is a \lstinline|double[][]|.
The constructor we use for this looks like:
\lstinputlisting[firstnumber=26,firstline=26,lastline=30]{"./assets/Stats.java"}

\end{document}