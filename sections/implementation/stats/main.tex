\documentclass[../Dissertation.tex]{subfiles}
\begin{document}
\subsection{Stats.java}
\label{section:stats}
The \lstinline|Stats.java| class contains a number of methods that perform the necessary statistical calculations for all our VaR measures.
This class is not intended to be used as an instance variable like \lstinline|Parameteres.java|, \lstinline|Results.java| or \lstinline|optionsData.java|.
Whereas these classes have \textit{Setters} for writing data, the methods in \lstinline|Stats.java| are essentially just functions - give it some data and it will return a result.

It does, however, contain some instance variables:
\lstinputlisting[firstnumber=10,firstline=10,lastline=15]{"./assets/Stats.java"}
Accompanying these are a number of constructors in which these instance variables get initialized.
The constructor that we use dictates the type of method we can call.
Now, the methods in \lstinline|Stats.java| are not consistent on what data types they require.
Some require just a \lstinline|double[]|.
Others require a \lstinline|double[][]| or a pair of \lstinline|double[]|s.
Either way, these methods are reliant upon the instance variables for their input variables.
As such, we need a variety of constructors to ensure that the right instance variables for our method are used.

%\subsubsection{getAbsoluteChanges()}

%In Backtesting, we need to be able to compare our VaR estimates with the real losses that our portfolio would have incurred at the time.
%We use \lstinline|getAbsoluteChanges()| which is a simple modification of \lstinline|getPercentageChanges|.
%Here, our calculation at every element is:
%\begin{lstlisting}[firstnumber=134,basicstyle=\fontsize{8}{10}\sffamily, escapeinside={@}{@}]
%priceDiff[i][j] = multiVector[i][j]- multiVector[i][j+1];
%\end{lstlisting}

\end{document}