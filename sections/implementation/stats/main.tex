\documentclass[../Dissertation.tex]{subfiles}
\begin{document}
\subsection{Stats.java}
\label{section:stats}
The \lstinline|Stats.java| class contains a number of methods that perform the necessary statistical calculations for all our VaR measures.
This class is not intended to be used as an instance variable like \lstinline|Parameteres.java|, \lstinline|Results.java| or \lstinline|optionsData.java|.
Whereas these classes have \textit{Setters} for writing data, the methods in \lstinline|Stats.java| are essentially just functions - give it some data and it will return a result.

It does, however, contain some instance variables:
\lstinputlisting[firstnumber=10,firstline=10,lastline=15]{"./assets/Stats.java"}
Accompanying these are a number of constructors in which these instance variables get initialized.
The constructor that we use dictates the type of method we can call.
Now, the methods in \lstinline|Stats.java| are not consistent on what data types they require.
Some require just a \lstinline|double[]|. 
Others require a \lstinline|double[][]| or a pair of \lstinline|double[]|s.
Either way, these methods are reliant upon the instance variables for their input variables.
As such, we need a variety of constructors to ensure that the right instance variables for our method are used.

\subsubsection{List of Methods}

The following is a list of public methods in \lstinline|Stats.java|. 
\begin{center}
\begin{tabular}{||c | c | c ||} 
\hline
Name & Constructor & Returns \\ [0.5ex] 
\hline\hline
\lstinline|getMean()| & \lstinline|double[]| & \lstinline|double|\\ 
\hline
\lstinline|getVariance(int measure)| & \lstinline|double[], double[]| & \lstinline|double| \\
\hline
\lstinline|getEWVariance()| & \lstinline|double[], double[]| & \lstinline|double| \\
\hline
\lstinline|getEWMAVariance()| & \lstinline|double[], double[]| & \lstinline|double|\\
\hline
\lstinline|getGARCH11Variance()| & \lstinline|double[], double[]| & \lstinline|double|\\
\hline
\lstinline|getVolatility(int measure)| & \lstinline|double[]| & \lstinline|double|\\
\hline
\lstinline|getEWVolatility()| & \lstinline|double[]| & \lstinline|double|\\
\hline
\lstinline|getEWMAVolatility()| & \lstinline|double[], double[]| & \lstinline|double|\\
\hline
\lstinline|getGARCH11Volatility()| & \lstinline|double[], double[]| & \lstinline|double|\\
\hline
\lstinline|getCorrelationMatrix(int measure)| & \lstinline|double[]| & \lstinline|double[][]|\\
\hline
\lstinline|getCovarianceMatrix()| & \lstinline|double[][]| & \lstinline|double[][]|\\
\hline
\lstinline|getCholeskyDecomposition()| & \lstinline|double[][]| & \lstinline|double[][]|\\
\hline
\lstinline|getPercentageChanges()| &\lstinline|double[][]| & \lstinline|double[][]|\\
\hline
\lstinline|getAbsoluteChanges()| & \lstinline|double[][]| & \lstinline|double[][]|\\
\hline
\lstinline|printMatrixToCSV()|* &\lstinline|double[][]| & \lstinline|void|\\
\hline
\lstinline|printVectorToCSV()|* & \lstinline|double[]| & \lstinline|void|\\[1ex] 
\hline
\end{tabular}
\end{center}
(*) While these methods do not perform statistical calculations, they do provide us with an easy way of printing data to \lstinline|csv|.
This is desirable if we want to analyse the distribution of our stock price changes produced from Monte Carlo simulation or look at the 1000 VaR estimates produced in Backtesting.
To implement these methods, we used information submitted by Tataje~\cite{Tataje:2013} and Pasini~\cite{Pasini:2016} on Stack Overflow on how to output data to \lstinline|csv| using \lstinline|BufferedWriter| and \lstinline|StringBuilder|.

\subsubsection{getMean()}

One of our main statistical assumptions is that, for a given vector of price changes $u_i$, its mean $\bar{u} = 0$.
Despite this, it is still useful to include a method that will return the mean of our array of price changes as a sanity check.
Indeed, this is how \lstinline|getMean()| is used in the class \lstinline|PortfolioInfo.java|, where we print a break-down of our portfolio.
\lstinputlisting[firstnumber=33,firstline=33,lastline=36]{"./assets/Stats.java"}
In our implementation of this method, we simply iterate through each element in \lstinline|double[] singleVector| and sum their values.
Then we return the average of this sum.

\subsubsection{getEWVariance()}
This method returns an equal-weighted estimation of daily variance.
It is our Java implementation of equation~\ref{eqn:covariancexy}, which describes the calculation of the equal-weighted covariance between two market variables.
We implement this equation (and not the one that describes variance) out of convenience.
We are able to use it for estimating \textit{both} variance and covariance.
This is because this method takes two \lstinline|double[]| variables as inputs.
In computing the variance, both inputs must represent the same market variable.
That is, the covariance of a market variable is simply the variance.
When computing covariance the two inputs should represent different market variables. 
\lstinputlisting[firstnumber=39,firstline=39,lastline=42]{"./assets/Stats.java"}
We simply iterate through each element in \lstinline|double[] xVector| and \lstinline|double[] yVector| simultaneously and sum their product of their values.
Then we return the average of this sum.

\subsubsection{getEWMAVariance()}

This method follows equation~\ref{eqn:covarianceEWMA} which is the EWMA estimate of daily-covariance. 
The reasoning here for implementing covariance and not simply variance is the same as in \lstinline|getVariance()|.
\begin{lstlisting}[firstnumber = 45]
        double lambda = 0.94;
        double EWMA = xVector[numRow -1] * yVector[numRow -1];	
\end{lstlisting}

Here, we use J.P. Morgan's RiskMetrics estimation of lambda $\lambda = 0.94$.
The computation of EWMA is recursive and iterative so we must define some initial estimate for variance to begin with.
We use the product of the two earliest data points.
We then iterate through the rest of the data, going forwards through time.
At each iteration, we update our EWMA estimate.
\begin{lstlisting}[firstnumber = 47,basicstyle=\fontsize{8}{10}\sffamily, escapeinside={@}{@}]
	for (int i = 1; i < numRow; i++)
            EWMA = lambda * EWMA @\newline@+ (1-lambda) * xVector[numRow -1 - i] * yVector[numRow -1 - i];
\end{lstlisting}
At the end of the loop, we return \lstinline|EWMA|.

\subsubsection{getGARCH11Variance()}

This is our implementation of the GARCH(1,1) estimate of daily-variance.
Just as in the previous two methods for estimating daily-variance, we take the covariance approach and follow equation~\ref{eqn:covarianceGARCH}.
GARCH(1,1) requires the estimation of three parameters, \lstinline|double omega, alpha, beta|.
These are found using the private method \lstinline|LevenbergMarquardt(double[] uSquaredArray)|, where \lstinline|uSquaredArray| is simply an array containing the product of the instance variables \lstinline|double[] xVector| and \lstinline|yVector|.
\begin{lstlisting}[firstnumber = 60,basicstyle=\fontsize{8}{10}\sffamily, escapeinside={@}{@}]
for (int i = 1; i < uSquared.length;i++)
            sigmaSquared = omega @\newline@+ (alpha*uSquared[i]) + (beta*sigmaSquared);
\end{lstlisting}            
Once these parameters are known, then the process of estimating the variance is similar to steps taken in \lstinline|getEWMAVariance()|.
We simply iterate through each observation, going forwards through time.
At each step, we update our estimation of the variance.
At the end of the loop, we return this estimate.

\paragraph{LevenbergMarquardt()}

The Levenberg-Marquardt algorithm is a hybrid of gradient descent and Newtonian method. 
We use it to maximise the log-likelihood function from equation~\ref{eqn:maxlikelihood}.
This log-likelihood is computed using the private method \lstinline|likelihood()|.

In order to estimate the three parameters for GARCH(1,1), we must define some initial estimates for omega, alpha and beta.
\begin{lstlisting}[firstnumber = 176]
	parameters[0] = 0.000001346;	//omega
	parameters[1] = 0.08339			//alpha
	parameters[2] = 0.9101;			//beta
\end{lstlisting}
These initial estimates were lifted out of Chapter 22, page 506 in Hull~\cite{Hull:2012}.
Choosing the right initial parameters is important: The algorithm is capable of making rapid descents, but it can get stuck easily in local valleys, instead of the global valley.

We also need to specify some small fudge factor parameter. 
To begin with, we specify \lstinline|private double lambda = 0.001;|
This parameter governs the size of our steps.
The algorithm adjusts this parameter accordingly depending on whether each step managed to increase the maximum likelihood estimate.
Because a number of methods need to be able to write to \lstinline|lambda|, it is an instance variable.

First we calculate return an estimation of the log-likelihood using the \lstinline|likelihood(uSquaredArray,parameters);| method.
Then we being a while loop which, on each iteration, maximises the log-likelihood until it can only produce insignificant increases and exits the loop.

In this loop, we produce some trial parameters using the method \lstinline|getTrialParameters(perameters,uSquaredArray)|.
Using these trial parameters, we compute a trial estimate of the maximum likelihood estimate.
If we have increased the likelihood, the trial parameters are accepted and we decrease the fudge factor by 10.
Otherwise, we ignore the trial parameters and revert to the previous parameters and increase the fudge factor by 10.

\paragraph{likelihood()}

This \lstinline|likelihood()| method is an implementation of the log-likelihood seen in equation~\ref{eqn:maxlikelihood}.
But first, we must generate an entire history of variance estimates by iterating through our data and computing the variance using a process similar to that found in the \lstinline|getGARCH11Variance()| method.
\begin{lstlisting}[firstnumber =205,basicstyle=\fontsize{8}{10}\sffamily, escapeinside={@}{@}]
for(int i = 1; i < variance.length; i++)
            variance[i] = omega + (alpha * uSquaredArray[i]) @\newline@+ (variance[i-1]* beta);
\end{lstlisting}
That is, we implement the formula from equation~\ref{eqn:covarianceGARCH}, this time populating each result into an array \lstinline|double[] variance| as we iterate through the data.
\begin{lstlisting}[firstnumber =205,basicstyle=\fontsize{8}{10}\sffamily, escapeinside={@}{@}]
for(int i = 0; i < variance.length; i++)
            likelihood += -Math.log(variance[i]) @\newline@- (uSquaredArray[i+1]/variance[i]);
\end{lstlisting}
Lastly, to compute the maximum likelihood estimate, we iterate through each of our variance estimates, summing the log-likelihood as we go.
At the end of the loop, we return \lstinline|likelihood|.

\paragraph{getTrialParameters()}

\lstinline|getTrialParameters()| takes \lstinline|double[] parameters| and \lstinline|double[] uSquaredArray| as inputs.
These being our current parameter estimates and the squared price changes respectively.
First we take a history of our variance estimates \lstinline|double[] variance| just as line \lstinline|likelihood()|.
Then, iterating through \lstinline|variance|, we compute the variables \lstinline|double dOmega, dAlpha, dBeta|.
\iffalse
\begin{lstlisting}[firstnumber = 231,basicstyle=\fontsize{8}{10}\sffamily, escapeinside={@}{@}]
	dOmega += ((-1/variance[i]) + (uSquaredArray[i]/Math.pow(variance[i],2)));@\newline@
	dAlpha += (-uSquaredArray[i]/variance[i])@\newline@+ (Math.pow(uSquaredArray[i],2)/Math.pow(variance[i],2));@\newline@
	dBeta 	+= (-variance[i-1]/variance[i])@\newline@+ ((uSquaredArray[i]*variance[i-1])/Math.pow(variance[i],2));
\end{lstlisting}
\fi
The computation of these variables are implementations of the partial differential equations seen in equations~\ref{eqn:partialomega},~\ref{eqn:partialalpha}, and~\ref{eqn:partialbeta}.

We then initialize the following array from equation~\ref{eqn:betavector}: 
\begin{lstlisting}[firstnumber = 236]
	double[] vectorBeta = {-0.5*dOmega, 0.5*dAlpha, -0.5*dBeta};
\end{lstlisting}

Then we enter a while-loop.
Firstly, we initialize the curvature matrix from equation~\ref{eqn:alphamatrix}:
\begin{lstlisting}[firstnumber=241,basicstyle=\fontsize{8}{10}\sffamily, escapeinside={@}{@}]
double[][] curvatureMatrix = {
{0.5*dOmega*dOmega * (1 + lambda), 0.5*dOmega*dAlpha, 0.5*dOmega*dBeta},@\newline@
{0.5*dAlpha*dOmega, 0.5*dAlpha*dAlpha * (1 + lambda), 0.5*dAlpha*dBeta},@\newline@
{0.5*dBeta*dOmega, 0.5*dBeta*dAlpha, 0.5*dBeta*dBeta * (1 + lambda)}};
\end{lstlisting}

Mathematically, we have a matrix and a vector, which is represented just as in equation~\ref{eqn:linear}.
In order to solve the simultaneous equations, we use the \lstinline|linear| package from the library \lstinline|org.apache.commons.math3|~\cite{Apache:Math3}.
The solution to the simultaneous equations is a an increment to the trial parameters.
If this increment is too small, then we break out of the while loop.
Otherwise, we increment the trial parameters.
The trial parameters are subject to a few conditions: 
				\begin{equation}
					\label{eqn:parameterconditions}
					a_j = \begin{cases}
						0 \leq \omega \leq \infty\\
						0 \leq \alpha \leq 1\\
						0 \leq \beta \leq 1-\alpha\\
						\alpha + \beta < 1
						\end{cases}
				\end{equation}
where $a_j$ is the parameter vector.
If the trial parameters break these conditions, we increase the fudge factor by a factor of ten and move on to the next iteration of the loop.
Otherwise, we will have found new trial parameters.
So we break the loop and return \lstinline|trialParameters|.

\subsubsection{Volatility methods}

In the univariate case, each volatility estimate is the square root of the estimate of daily variance.
We have three methods that return volatility estimates: \lstinline|getEWVolatility()|, \lstinline|getEWMAVolatility()|, and \lstinline|getGARCH11Volatility()|.
Implementing each of these methods is simple.
\begin{lstlisting}[firstnumber = 65,basicstyle=\fontsize{8}{10}\sffamily, escapeinside={@}{@}]
    public double getEWVolatility(){return Math.sqrt(getEWVariance());}
    public double getEWMAVolatility(){return Math.sqrt(getEWMAVariance());}
    public double getGARCH11Volatility(){return Math.sqrt(getGARCH11Variance());}
\end{lstlisting}
We call a method for the corresponding estimate of daily-variance and take the square root.

\subsubsection{Encapsulating Variance and Volatility}

Because we have so many methods for computing variance and volatility, it can be difficult to write code that isn't cumbersome.
For instance, suppose we have a single array of price changes.
We want to use a loop to compute all three variance estimates, but we cannot.
We have to instead avoid a loop and write repetitive code that calls each method one by one.

To avoid this, we have implemented some helper methods.
These take an input variable \lstinline|int measures|, in which certain values are encoded to correspond to another method.
The integers 1, 2, 3 encode to \textit{Equal-Weighted}, \textit{EWMA},  and  \textit{GARCH(1,1)} respectively.

We have two methods in this case: \lstinline|getVariance(int measure)| and \lstinline|getVolatility(int measure)|.
For example, the following code will return an Equal-Weighted variance estimate of our price changes.
\begin{center}
	\lstinline|double variance = new Stats(priceChanges, priceChanges).getVariance(1);|
\end{center}
The next example will return the GARCH(1,1) volatility estimate:
\begin{center}
	\lstinline|double volatility = new Stats(priceChanges, pricechanges).getVolatility(3);|
\end{center}	
This is a particularly useful feature when it comes to our implementation of \lstinline|getCorrelationMatrix()|.

\subsubsection{getCorrelationMatrix()}

 We are able to construct correlation matrices using any of our variance and volatility methods.
 Thanks to the encapsulation discussed above, we are able write quite general code that still allows us to switch between the types of estimates very easily.
 First we declare a two-dimension square matrix. 
 \begin{center}
 	\lstinline|double[][] matrix = new double[numCol][numCol];|
 \end{center}
 It has a length equal to the number of stock variables in our portfolio.
 Using a nested loop, we iterate through each element and compute the correlation as in equation~\ref{eqn:correlation}.
 
That is, at each element we compute the covariance between the $i$th and $j$th variable using the method \lstinline|getVariance(measure)|, where \lstinline|measure| is an integer that simply specifies what sort of estimate we want (EW, EMWA or GARCH(1,1).
At the same time, with the \lstinline|getVolatility(measure)| method, we compute the volatility estimate for the $i$th variable and the same for the $j$th variable.
\begin{lstlisting}[firstnumber = 94]
	matrix[i][j] = covXY / (sigmaX * sigmaY);
\end{lstlisting}
Then we take the covariance estimate and divide it by the product of our two volatility estimates to compute the correlation estimate for that element in the matrix.
Once we populate the entire matrix, we return \lstinline|matrix|.


\subsubsection{getCovarianceMatrix()}

Building a variance-covariance matrix is similar to building the correlation matrix - especially since covariance and correlation are so mathematically intertwined.
As before, we declare a two dimensional matrix
\begin{lstlisting}[firstnumber = 101]
 	double[][] covarianceMatrix = new double[numCol][numCol];
\end{lstlisting}
 We iterate through each element in the same way using a nested loop and populate the value of each element by calling the \lstinline|getVariance(measure)| method.
 \begin{lstlisting}[firstnumber=104,basicstyle=\fontsize{8}{10}\sffamily, escapeinside={@}{@}]
 covarianceMatrix[i][j] @\newline@= new Stats(multiVector[i], multiVector[j]).getVariance(measure);
 \end{lstlisting}
 When the entire matrix is populated, we return \lstinline|covarianceMatrix|.


\subsubsection{getCholeskyDecomposition()}

This method is an implementation of the Cholesky Decomposition, following equations~\ref{eqn:diagonalL} and~\ref{eqn:belowdiagonalL}.
In our Monte Carlo simulation, we are trying to find $L$ such that $\Sigma = LL^T$ where $\Sigma$ is our variance-covariance matrix.

This method does not take a covariance matrix as an input. 
That is because in \lstinline|MonteCarlo.java|, we do not need one and as such never create one.
Therefore, we first build a covariance matrix using the method \lstinline|getCovarianceMatrix(measure)|.
Then we initialize a matrix for $L$.
\begin{lstlisting}[firstnumber=108,basicstyle=\fontsize{8}{10}\sffamily, escapeinside={@}{@}]
		double[][] covarianceMatrix = getCovarianceMatrix(measure);
		double[][] cholesky@\newline@ = new double[covarianceMatrix.length][covarianceMatrix.length];
\end{lstlisting}

Then we enter a nested loop and iterate through each of the remaining elements.
At every element, we do the following:
\begin{lstlisting}[firstnumber=113,basicstyle=\fontsize{8}{10}\sffamily, escapeinside={@}{@}]
                double sum = 0;
                for (int k = 0; k < j; k++)
                    sum += cholesky[i][k] * cholesky[j][k];
                if(i==j)
                    cholesky[i][j] = Math.sqrt(covarianceMatrix[i][j] - sum);
                else
                    cholesky[i][j] = (covarianceMatrix[i][j] - sum) / cholesky[j][j];
\end{lstlisting}
That is, we compute \lstinline|sum| as $\sum_{k=1}^{i-1}L_{ik}L_{jk}$.
Then, if we are on a diagonal, we subtract \lstinline|sum| from the value on the same element on the covariance matrix and take the square root.
If we are \textit{under} the diagonal, we don't take the square root but divide by the diagonal element at $i$.
When we exit the loop, we return the matrix \lstinline|cholesky|.

\subsubsection{getPercentageChanges()}

Underpinning all our calculations when estimating VaR is our historical data.
We need to compute the daily percentage changes, as seen in equation~\ref{eqn:returns} when using any VaR measure.
The method \lstinline|getPercentageChanges()| will take a two-dimensional array of stock prices and iterate first by asset and then by day.
Or, in other words, we use a nested loop in which we iterate through every element in a column and then move onto the next column.
At each element we compute:
\begin{lstlisting}[firstnumber=127,basicstyle=\fontsize{8}{10}\sffamily, escapeinside={@}{@}]
	priceDiff[i][j] = ((multiVector[i][j]- multiVector[i][j+1])@\newline@/ multiVector[i][j+1]);
\end{lstlisting}
The end result is the two-dimensional array \lstinline|double[][] priceDiff| that will be one row shorter than our array of stock data.
If we call \lstinline|getMean()| on \lstinline|priceDiff|, we will return a result very close to zero.

%\subsubsection{getAbsoluteChanges()}

%In Backtesting, we need to be able to compare our VaR estimates with the real losses that our portfolio would have incurred at the time.
%We use \lstinline|getAbsoluteChanges()| which is a simple modification of \lstinline|getPercentageChanges|.
%Here, our calculation at every element is:
%\begin{lstlisting}[firstnumber=134,basicstyle=\fontsize{8}{10}\sffamily, escapeinside={@}{@}]
%priceDiff[i][j] = multiVector[i][j]- multiVector[i][j+1];
%\end{lstlisting}

\end{document}