\documentclass[../Dissertation.tex]{subfiles}
\begin{document}
\subsubsection{getCovarianceMatrix()}

Building a variance-covariance matrix is similar to building the correlation matrix - especially since covariance and correlation are so mathematically intertwined.
As before, we declare a two dimensional matrix
\begin{lstlisting}[firstnumber = 101]
 	double[][] covarianceMatrix = new double[numCol][numCol];
\end{lstlisting}
We iterate through each element in the same way using a nested loop and populate the value of each element by calling the \lstinline|getVariance(measure)| method.
\begin{lstlisting}[firstnumber=104,basicstyle=\fontsize{8}{10}\sffamily, escapeinside={@}{@}]
 covarianceMatrix[i][j] @\newline@= new Stats(multiVector[i], multiVector[j]).getVariance(measure);
 \end{lstlisting}
When the entire matrix is populated, we return \lstinline|covarianceMatrix|.

\end{document}