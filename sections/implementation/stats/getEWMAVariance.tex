\documentclass[../Dissertation.tex]{subfiles}
\begin{document}
\subsubsection{getEWMAVariance()}

This method follows equation~\ref{eqn:covarianceEWMA} which is the EWMA estimate of daily-covariance.
The reasoning here for implementing covariance and not simply variance is the same as in \lstinline|getVariance()|.
\begin{lstlisting}[firstnumber = 45]
        double lambda = 0.94;
        double EWMA = xVector[numRow -1] * yVector[numRow -1];	
\end{lstlisting}

Here, we use J.P. Morgan's RiskMetrics estimation of lambda $\lambda = 0.94$.
The computation of EWMA is recursive and iterative so we must define some initial estimate for variance to begin with.
We use the product of the two earliest data points.
We then iterate through the rest of the data, going forwards through time.
At each iteration, we update our EWMA estimate.
\begin{lstlisting}[firstnumber = 47,basicstyle=\fontsize{8}{10}\sffamily, escapeinside={@}{@}]
	for (int i = 1; i < numRow; i++)
            EWMA = lambda * EWMA @\newline@+ (1-lambda) * xVector[numRow -1 - i] * yVector[numRow -1 - i];
\end{lstlisting}
At the end of the loop, we return \lstinline|EWMA|.
\end{document}