\documentclass[../Dissertation.tex]{subfiles}
\begin{document}
\subsubsection{getPercentageChanges()}

Underpinning all our calculations when estimating VaR is our historical data.
We need to compute the daily percentage changes, as seen in equation~\ref{eqn:returns} when using any VaR measure.
The method \lstinline|getPercentageChanges()| will take a two-dimensional array of stock prices and iterate first by asset and then by day.
Or, in other words, we use a nested loop in which we iterate through every element in a column and then move onto the next column.
At each element we compute:
\begin{lstlisting}[firstnumber=127,basicstyle=\fontsize{8}{10}\sffamily, escapeinside={@}{@}]
	priceDiff[i][j] = ((multiVector[i][j]- multiVector[i][j+1])@\newline@/ multiVector[i][j+1]);
\end{lstlisting}
The end result is the two-dimensional array \lstinline|double[][] priceDiff| that will be one row shorter than our array of stock data.
If we call \lstinline|getMean()| on \lstinline|priceDiff|, we will return a result very close to zero.
\end{document}