\documentclass[../Dissertation.tex]{subfiles}
\begin{document}
\subsubsection{getCorrelationMatrix()}

We are able to construct correlation matrices using any of our variance and volatility methods.
Thanks to the encapsulation discussed above, we are able write quite general code that still allows us to switch between the types of estimates very easily.
First we declare a two-dimension square matrix.
\begin{center}
	\lstinline|double[][] matrix = new double[numCol][numCol];|
\end{center}
It has a length equal to the number of stock variables in our portfolio.
Using a nested loop, we iterate through each element and compute the correlation as in equation~\ref{eqn:correlation}.

That is, at each element we compute the covariance between the $i$th and $j$th variable using the method \lstinline|getVariance(measure)|, where \lstinline|measure| is an integer that simply specifies what sort of estimate we want (EW, EMWA or GARCH(1,1).
At the same time, with the \lstinline|getVolatility(measure)| method, we compute the volatility estimate for the $i$th variable and the same for the $j$th variable.
\begin{lstlisting}[firstnumber = 94]
	matrix[i][j] = covXY / (sigmaX * sigmaY);
\end{lstlisting}
Then we take the covariance estimate and divide it by the product of our two volatility estimates to compute the correlation estimate for that element in the matrix.
Once we populate the entire matrix, we return \lstinline|matrix|.
\end{document}