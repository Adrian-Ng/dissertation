\documentclass[../Dissertation.tex]{subfiles}
\begin{document}
\subsubsection{Volatility methods}

In the univariate case, each volatility estimate is the square root of the estimate of daily variance.
We have three methods that return volatility estimates: \lstinline|getEWVolatility()|, \lstinline|getEWMAVolatility()|, and \lstinline|getGARCH11Volatility()|.
Implementing each of these methods is simple.
\begin{lstlisting}[firstnumber = 65,basicstyle=\fontsize{8}{10}\sffamily, escapeinside={@}{@}]
    public double getEWVolatility(){return Math.sqrt(getEWVariance());}
    public double getEWMAVolatility(){return Math.sqrt(getEWMAVariance());}
    public double getGARCH11Volatility(){return Math.sqrt(getGARCH11Variance());}
\end{lstlisting}
We call a method for the corresponding estimate of daily-variance and take the square root.

\subsubsection{Encapsulating Variance and Volatility}

Because we have so many methods for computing variance and volatility, it can be difficult to write code that isn't cumbersome.
For instance, suppose we have a single array of price changes.
We want to use a loop to compute all three variance estimates, but we cannot.
We have to instead avoid a loop and write repetitive code that calls each method one by one.

To avoid this, we have implemented some helper methods.
These take an input variable \lstinline|int measures|, in which certain values are encoded to correspond to another method.
The integers 1, 2, 3 encode to \textit{Equal-Weighted}, \textit{EWMA},  and  \textit{GARCH(1,1)} respectively.

We have two methods in this case: \lstinline|getVariance(int measure)| and \lstinline|getVolatility(int measure)|.
For example, the following code will return an Equal-Weighted variance estimate of our price changes.
\begin{center}
	\lstinline|double variance = new Stats(priceChanges, priceChanges).getVariance(1);|
\end{center}
The next example will return the GARCH(1,1) volatility estimate:
\begin{center}
	\lstinline|double volatility = new Stats(priceChanges, pricechanges).getVolatility(3);|
\end{center}
This is a particularly useful feature when it comes to our implementation of \lstinline|getCorrelationMatrix()|.
\end{document}