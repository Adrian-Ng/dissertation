\documentclass[../Dissertation.tex]{subfiles}
\begin{document}
\subsubsection{Standard Coverage Test}
\label{section:coveragestandard}
Let us formally define what constitutes a violation via the excedence process $I$:
				\begin{equation}
					\label{eqn:excedenceprocess}
					I^t = \begin{cases}
						0 \text{ if loss } \Pi^{t-1}-\Pi^t \leq VaR\\
						1 \text{ if loss } \Pi^{t-1}-\Pi^t > VaR\\
						\end{cases}
				\end{equation}
The number of violations observed in the data is then:
				\begin{equation}
					\label{eqn:numviolations}
					v = \sum_{t=0}^\alpha I^t
				\end{equation}
we take $v$ as the realization of the random variable $V$ which takes the binomial distribution:
				\begin{equation}
					\label{eqn:binomial}
					V\sim B(\alpha+1,1-c)
				\end{equation}
where $\alpha$ is the number of moments of VaR we take in our backtesting.

Suppose we have some significance level $\epsilon$, which controls the precision of our estimation.
We are able to test $\mathit{H_0}$ at any $\epsilon$.
To do this, we must determine the interval $[v_1,v_2]$ such that:
				\begin{equation}
					\label{eqn:interval}
					\mathbb{P}(V\notin [v_1,v_2])\leq \epsilon
				\end{equation}
is maximized.
We also want an interval that is \textit{generally} symmetric, i.e.:
				\begin{equation} 
					\label{eqn:symmetry}
					\mathbb{P}(V<v_1)\approx\mathbb{P}(v_2<V)\approx\epsilon/2
				\end{equation}			
Subject to these constraints, we optimize the parameter $n$ that maximises equation~\ref{eqn:interval} having defined our interval as either:
				\begin{equation}
						\label{eqn:intervaloptimize}
						\lbrack a + n,b\rbrack\text{ or }\lbrack a, b-n\rbrack
				\end{equation}
where:				
\begin{conditions}	
$n$ & non-negative integer\\
$a$ & maximum integer such that $\mathbb{P}(V<a)\leq \epsilon/2$\\
$b$ & minimum integer such that $\mathbb{P}(b<V)\leq \epsilon/2$\\
\end{conditions}
Initially $n\gets0$, so $a$ and $b$ are starting parameters for our interval.
As we optimize, we increment $n$ until our conditions are met.

The result of this is a non-rejection interval in which $v$ is permissible.
If, however, $v$ is outside this interval, we must reject our VaR measure at our significance level $\epsilon$.

The confidence level is a measure of how confident we are in our portfoilio.
The significance level is a measure of how precise we want our estimate to be.
It dictates the width of our interval.
Experimentally, we can begin with a significance level that governs a wide non-rejection level and decrease until we start to see rejections.

\end{document}