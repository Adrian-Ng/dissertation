\documentclass[../background_research.tex]{subfiles}
\begin{document}
\subsubsection{Standard Deviation}

The estimate of the daily volatility is calculated as the square root of the \textit{variance-rate} $\sigma^2_{n}$ on day $n$:
\begin{equation}
    \label{eqn:variance}
    \sigma^2_{n} = \frac{1}{m-1}\sum_{i=1}^m{(u_{n-i}-\bar{u})^2}
\end{equation}
Simply, daily volatility is the standard deviation of daily returns.
In equation~\ref{eqn:variance}, we take $\frac{1}{m-1}$ so that our estimate is unbiased.
However, for simplicity we take $\frac{1}{m}$ instead.
Now we can simplify our equation and estimate the squared volatility as:
\begin{equation}
    \label{eqn:varianceSimplified}
    \sigma^2_n={\frac{1}{m}}\sum_{i=1}^mu^2_{n-i}
\end{equation}
In this equation, we give equal weight to each value of $u^2_{n-i}$.
This type of model is fine if volatility is constant; this is an acceptable assumption for small periods of time.
But we cannot make this assumption for long periods of time.
If we compare last year's volatility estimate to today's, we can expect it would be different.

When estimating volatility, it makes sense to assume instead that more recent price changes are more relevant than those in the past.
But on the other hand, we naturally want to include in our calculations as many observations as possible to produce the best estimate.
This is a trade-off that needs to be reconciled.

Let us consider, as Hull suggests, the following example - a simple weighted model:
\begin{equation}
    \label{eqn:weightedmodel}
    \sigma^2 _n = \sum_{i=1}^m{\alpha_i}u^2_{n-i}
\end{equation}
Here, we give observations from $i$ days ago a weight of $\alpha_i$.
An observation $j$ days ago is more recent than an observation $i$ days ago, (i.e. $i > j$).
We then assign to each observation a weight such that $a_i < a_j$.
More recent observations $u_j$ have higher weighting than older observations $u_i$.
Note that the $\alpha$'s must be positive and sum to unity such that $\sum_{i=1}^m{\alpha_i} = 1$.

\end{document}