\documentclass[../background_research.tex]{subfiles}
\begin{document}
\subsubsection{Estimating Covariance}

Now, we have discussed how to estimate the daily volatilities $\sigma_x$ and $\sigma_y$.
If we calculate the vector of price changes $x_i$ for stock $X$ and the vector of price changes $y_i$ for stock $Y$ using the same method in equation~\ref{eqn:returns} then we can assume $\bar{x}=\bar{y} = 0$.
As such, we can then use equation~\ref{eqn:varianceSimplified} to estimate the \textit{variance-rate}.
\begin{equation}
    \label{eqn:variancexy}
    \sigma^2_{x,n}={\frac{1}{m}}\sum_{i=1}^mx^2_{n-i},\quad \sigma^2_{y,n}={\frac{1}{m}}\sum_{i=1}^my^2_{n-i}
\end{equation}
Our estimate for the covariance on day $n$ between $X$ and $Y$ is calculated as:
\begin{equation}
    \label{eqn:covariancexy}
    \text{cov}_n={\frac{1}{m}}\sum_{i=1}^mx_{n-i}y_{n-i}
\end{equation}
As you can see this is similar to our estimate of $\sigma^2$.

Likewise, if we are attempting the EWMA approach, then our covariance estimate for day $n$ takes a similar form to our estimation of the variance-rate in equation~\ref{eqn:ewma1}:
\begin{equation}
    \label{eqn:covarianceEWMA}
    \text{cov}_n= \lambda{\text{cov}_{n-1}} + (1-\lambda)x_{n-1}y_{n-1}
\end{equation}
Again, we see a recursive approach for our covariance estimation.
If we were to perform another substitution analysis, we would also see that the weights given to the observations decline at the exponential rate the further we look into the past.

For GARCH(1,1), covariance is estimated as:
\begin{equation}
    \label{eqn:covarianceGARCH}
    \text{cov}_n=\omega+\alpha{x_{n-1}}y_{n-1}+\beta\text{cov}_{n-1}
\end{equation}
where our parameters $\omega, \alpha$, and $\beta$ can be found by maximising the following:
\begin{equation}
    \label{eqn:maxlikelihood2}
    \sum_{i=1}^m\Bigg[-\ln({\text{cov}_i}) - \frac{x_i{y_i}}{\text{cov}_i}\Bigg]
\end{equation}

Once we have our covariance estimates, we are able to produce a \textit{variance-covariance} matrix.

\end{document}