\documentclass[../Dissertation.tex]{subfiles}
\begin{document}
\section{Background Research}
\label{section:background}

\subsection{Estimating Variance and Volatility}

In this section our treatment follows Hull \cite{Hull:2012} chapter 22, page 498 onwards.

Suppose we have a vector of stock prices ${S_i}$.
Each element in this vector represents a daily reading of stock prices over some fixed interval $\tau$.
This is our historical data.
We assume it takes a normal distribution.

In order to calculate daily volatilities from this data, we must iterate through this vector and calculate a vector of percentage changes (=returns).
Hull \cite{Hull:2012} defines this via the following equation:
\begin{equation}
    \label{eqn:returns}
    {u_i} = \frac{{S_i}-{S_{i-1}}}{S_{i-1}}
\end{equation}
This is the percentage change ${u_i}$ between the end of day ${i-1}$ and the end of day ${i}$.
In calculating the percentage change this way, we can assume the mean of ${u_i}$ given by $\bar{u}={\frac{1}{m}\sum_{i=1}^m{u_{n-i}}}$, is approximately zero such that we simply assume it is zero.



\end{document}