\documentclass[../analytical_approach.tex]{subfiles}
\begin{document}
\subsubsection{Single Asset}

We can model the behaviour of the stock price over some time interval $dt$ via the Stochastic equation:
\begin{equation}
    \label{eqn:stochastic}
    \frac{dS}{S}=\mu dt + \sigma dz
\end{equation}
where $dz = \epsilon\sqrt{dt}$. Assuming some small (but finite) interval of time $\Delta t$, we can rewrite this as:
\begin{equation}
    \label{eqn:stochastic2}
    \frac{dS}{S}=\mu \Delta t + \sigma \epsilon\sqrt{dt}
\end{equation}
where $\epsilon\sim\phi{0,1}$.
We can simplify this equation by ignoring $\mu \Delta t$, which is relatively small over small time intervals.
This is because, in practice, the interest rate $\mu$ is much smaller than the volatility $\sigma$.
Suppose $\mu = 0.1$ and $\sigma = 0.35$ and we consider a day over a yearly interval, so $\Delta t = \frac{1}{252}$, then $\mu \Delta t \approx 0.0004$ and $\sigma \sqrt{\Delta t} \approx 0.022$.
We see now that the latter dominates the former and at small time intervals $\mu \Delta t$ is small enough to be ignored.
So simplifying:
\begin{equation}
    \label{eqn:stochastic3}
    \frac{dS}{S}=\sigma \epsilon\sqrt{dt}
\end{equation}
As such we can take the approximation of the change of our share price to be Normally distributed:
\begin{equation}
    \label{eqn:distributionS}
    \Delta S\sim\phi(0,S^2\cdot\sigma^2\Delta t)
\end{equation}
If our portfolio consists of $k$ shares then the value of the portfolio is $\Pi=kS$.
Its change in value will therefore be $\Delta \Pi = k \Delta S$.
Likewise, we multiply the variance of $\Delta S$ by $k^2$ such that the distribution of $\Delta\Pi$ is:
\begin{equation}
    \label{eqn:distributionPi}
    \Delta \Pi\sim\phi(0,k^2S^2\cdot\sigma^2\Delta t)
\end{equation}

Value at Risk takes two parameters: time horizon and confidence level. Suppose we take the confidence level $c=99\%$.
This means we are $99\%$ sure that we won't lose more than $V$, our estimate of Value at Risk.
That is, the change in the value of our portfolio is $\Delta\Pi<-V$.
Since we want the probability of this event to be $100-c=1\%$ or less, we need to solve this equation:
\begin{equation}
    \label{eqn:probabilityV}
    \mathbb{P}(\Delta\Pi<-V) = 1\%
\end{equation}
How, then, do we calculate $V$?
$\Pi$ is Gaussian with a mean of zero.
The density of the change in the value of the portfolio looks like:\\\\
\begin{figure}
    \centering
    \caption{Distribution of $\Delta\Pi$}
    \label{fig:deltapi}
    \begin{tikzpicture}
        \begin{axis}[
                no markers
                ,	domain=-3:3
                ,	samples=100
                ,	axis lines*=left
                ,	xlabel=$\Delta\Pi$
                %,	ylabel=$y$
                %,	every axis y label/.style={at=(current axis.above origin),anchor=south}
                ,	hide y axis
                ,	every axis x label/.style={at=(current axis.right of origin),anchor=west}
                ,	height=5cm, width=12cm
                ,	xtick={0}
                , 	ytick=\empty
                ,	xticklabels={0}
                ,	enlargelimits=false
                ,	clip=false
                ,	axis on top
                ,	grid = major
            ]
            \addplot [fill=orange!20, draw=none, domain=-3:-2.33] {gauss(0,1)} \closedcycle;
            \addplot [very thick,cyan!50!black] {gauss(0,1)};
            \draw[dashed] (axis cs:-2.33,0)node [below]{$-V$} -- (axis cs:-2.33,0.05) node [above]{$1\%$};
        \end{axis}
    \end{tikzpicture}
\end{figure}
We need to find $V$ such that the shaded area is $1\%$.
Analytically, we look at this problem in terms of standard Gaussian.
We can define $\Delta\Pi$:
\begin{equation}
    \label{eqn:deltapi}
    \Delta\Pi = \epsilon\Pi\sigma\sqrt{\Delta t}
\end{equation}
where $\epsilon$ is the standard Gaussian.
If we substitute this into equation~\ref{eqn:probabilityV} we get:
\begin{equation}
    \mathbb{P}(\epsilon\Pi\sigma\sqrt{\Delta t}<-V)=1\%
\end{equation}
Rearranging, we no longer write the question in terms of $\Delta\Pi$ but in terms of the standard Gaussian:
\begin{equation}
    \label{eqn:definepercentile}
    \mathbb{P}\bigg(\epsilon<-\frac{V}{\Pi\sigma\sqrt{\Delta t}}\bigg)=1\%
\end{equation}
We define the percentile $x_{1\%}$ so that $\mathbb{P}(\epsilon\leq x_{1\%}) = 1\%$.
This number can be found by passing our significance level to the Gaussian inverse cumulative distribution function.
Once we have found this number we can equate it to:
\begin{equation}
    \label{eqn:percentile}
    x_{1\%}=-\frac{V}{\Pi\sigma\sqrt{\Delta t}}
\end{equation}
Thus we derive an exact formula for Value at Risk and find $V$:
\begin{equation}
    \label{eqn:varsinglestock}
    V =-x_{1\%}\Pi\sigma\sqrt{\Delta t}
\end{equation}

Wilmott~\cite{Wilmott:2007} writes it differently:
\begin{equation}
    \label{eqn:varsinglestock2}
    V =-\sigma\Delta{S}({\delta}{t})^{-1/2}\alpha(1-c)
\end{equation}
where $\alpha$ is the inverse cumulative distribution function for the standard Gaussian and the Greek letter $\Delta$ represents the number of stocks at price $S$.
The two equations are equivalent since $\Delta S = \Pi$.
But in this way we can explicitly see our parameters for VaR: confidence level $c$ and time horizon $\delta t$.

\end{document}