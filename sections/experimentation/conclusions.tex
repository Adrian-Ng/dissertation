\documentclass[../Dissertation.tex]{subfiles}
\begin{document}
\subsection{Conclusions}

In our experiments, we have had some successes and on the other hand, some questions remain unanswered.

Firstly, we were able to demonstrate the benefits of diversification.
In Section~\ref{sec:jointpositions} on page~\pageref{sec:jointpositions} onwards, we talk about how uncorrelated market assets can reduce the VaR in the portfolio.
As such we saw that the proportion of VaR decreases as we increase the number of assets in the portfolio.

With regards to the time horizon, we saw more violations on the whole with a 10-day time horizon.
This is in line with our expectations.

Some of results, on the other hand defy expectations.
For instance, we saw more violations in our non-rejection intervals during the Stress Test.
Now, is our implementation at fault or is the market at the current point in time \textit{stressed}?
One factor to take into consideration is that we did not have any options in our portfolios during the Stress Test.

The Monte Carlo and Analytical approaches generally performed well, but their results were expected to be similar.
Perhaps the number of random walks plotted was insufficient.
However, the Monte Carlo simulation is very slow.
At 10,000 random walks, it took us seven hours on a \textit{Core i5} processor to run the entire battery of experiments.
Unfortunately, increasing the number of random walks was not viable, given the time constraints.

Despite this, the Monte Carlo and Analytical EWMA performed well.
It is interesting to note that the lambda parameter that EWMA relies upon was taken from J.P. Morgan's RiskMetrics.
Whereas parameters for GARCH(1,1) were estimated by our own implementation of Levenberg-Marquardt and we saw strong inconsistencies in its performance.

The equal-weighted measures performed the worst
It could be argued that the equal-weighted assumption does not work.
And as for the Historical approach, which makes no probabilistic assumptions, but assumes that future market performance will be like the past, there is not much to say about it.
It is consistently in between the other measures.

Some questions which remain unanswered pertain to confidence and significance level.
We saw, in once instance, that we had zero rejections with a 1-day time horizon, 95\% confidence and 5\% significance levels during the Stress Test.
But ultimately, we still are not able to say how confident and how accurate we are willing to be with regard to significance and confidence levels.
\end{document}