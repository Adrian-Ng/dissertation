\documentclass[../Dissertation.tex]{subfiles}
\begin{document}
\section{Self-Assessment}

\paragraph{Planning}

The Gantt chart I produced was an overly optimistic plan for how everything would happen on a consistent, week-by-week basis.
It was of course, far from reality.
Instead of implementing each feature in piecemeal blocks of time, it was a process of continual improvement and refinement in all areas.
For instance, I expected to have data acquisition methods over and done with early on in the project.
Instead, I found myself revisiting different aspects of it again and again.
Options data was not fully implemented until mid-July, give or take.

I suffered unforeseen set-backs as well; I fell ill \textit{twice}, during which time no work was done.
Ultimately, while I didn't manage to stick to the plan in any chronological sense, I did manage to stick with the \textit{spirit} of the plan.
That is, nearly all the features written on there are implemented - except for the $\chi^2$ distributions.
\paragraph{Theory and Application}
Mathematics is not my expertise and has not been present in my life since my undergraduate studies some time ago.
The main challenge of this project was understanding the theory and applying it to the real-world, which is not very straight forward.
Just as there are many ways of estimating variance, I have learnt that there are many ways of approaching a problem.

So when it came to writing Java, I found that it was expected that I could potentially spend a lot of time trying one approach only to find that it was a dead end.
For instance, when I spent a few days looking into Yahoo Finance only to discover that the API had been shut down a month prior.
Or, for example, when I tried following Hull's~\cite{Hull:2012} approach to taking 501 days of data and found that bank holidays are a mystery unto themselves.
\paragraph{The Future}
Ultimately, I have enjoyed this project.
I hope it will lead me into a career in finance as a Data Scientist.
\end{document}