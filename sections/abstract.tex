\documentclass[../Dissertation.tex]{subfiles}
\begin{document}
\begin{abstract}
    Value at Risk (VaR) is an estimate which describes, in a single figure, the risk associated with our portfolio of investments.
    This dissertation attempts to demonstrate the implementation of various measures of VaR using Java.
    We discuss the theory of the \text{Analytical}, \text{Monte Carlo} and \text{Historical} approaches and detail the differences between these approaches.
    Some may take probabilistic assumptions and/or sample from simulated distributions.
    In all cases, we take the parameters of our approaches from real-world historical market data.

    We also look at the various approaches to estimating daily variance and volatility.
    These are the \textit{Equal-Weighed} (EW), the \textit{Exponentially Weighted Moving Average} (EWMA), the \textit{Generalised Autoregressive Conditional Heteroskedastic} (GARCH(1,1)) approaches.
    In the case of GARCH(1,1), we demonstrate parameter estimation via maximum likelihood estimation using the Levenberg-Marquardt algorithm.
    For EWMA, we take the parameter used by J.P. Morgan's RiskMetrics.

    We compare these measures against the real life performance in the stock market via back testing and stress testing.
    In stress testing, we use real-world data taken from a period of \textit{stressed} market conditions, starting from 1st January 2007.
    Our experiments will show that some measures, such as EWMA, perform well.
    Whereas other measures, like EW, do not.
    We also experiment with different portfolio configurations involving a number of stocks and hedging with put options.
    We show that we can reduce our exposure to risk by diversifying our portfolio.

    We source our data from Google Finance and demonstrate in our Java implementation the approach to acquisition of stock and options data.
    Additionally, we demonstrate an object-oriented approach to computing our statistics.

\end{abstract}
\end{document}